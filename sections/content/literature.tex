\chapter{Literature Survey}

\section{Introduction}
\label{sec:lit_review_intro}

This chapter provides a comprehensive review of the existing literature pertinent to the detection of advanced malware threats, specifically focusing on fileless and polymorphic variants. As established in the previous chapter, these types of malware pose significant challenges to traditional security mechanisms due to their evasive nature. Understanding the current state-of-the-art, including established techniques, recent advancements, and inherent limitations, is crucial for contextualizing the contributions of this thesis. The following sections will delve into key areas of research that form the foundation for the methodology proposed in this work. We will survey foundational and contemporary studies related to:
\begin{itemize}
    \item \textbf{Memory Forensics for Malware Detection:} Exploring techniques that analyze volatile memory to uncover malicious activities that evade disk-based detection.
    \item \textbf{Vision-Based Approaches to Malware Analysis:} Examining methods that leverage computer vision principles by visualizing binary data (including memory dumps) as images to identify malware patterns.
    \item \textbf{Feature Extraction Techniques:} Discussing specific image feature descriptors, such as GIST and HOG, and their application in representing malware characteristics visually.
    \item \textbf{Machine Learning Applications:} Reviewing how various machine learning algorithms have been employed to classify malware based on features extracted from memory or visual representations, particularly for challenging fileless and polymorphic types.
    \item \textbf{Dimensionality Reduction and Manifold Learning:} Investigating the use of techniques like UMAP to handle high-dimensional feature spaces and improve the detection of unknown or zero-day malware.
\end{itemize}

By critically examining these related fields, this review aims to identify the strengths and weaknesses of existing approaches. This analysis will culminate in the identification of specific research gaps and unresolved challenges, thereby highlighting the necessity and novelty of the adaptive detection framework developed in this thesis. This chapter serves not only as a survey of prior work but also as justification for the research objectives and methodology detailed subsequently.

\section{Existing Approaches}
\label{sec:existing_approaches}

This section reviews existing methodologies and research pertinent to the detection of fileless and polymorphic malware. The review is structured thematically, covering key technological areas that underpin the approach taken in this thesis. We begin by examining the critical role of memory forensics.

\subsection{Memory Forensics for Malware Detection}
\label{subsec:mem_forensics}

As malware increasingly employs techniques to evade disk-based detection, analysing volatile memory has become an indispensable tool in cybersecurity \cite{kara2022fileless}. Memory forensics involves the acquisition and analysis of a system's Random Access Memory (RAM) content to uncover evidence of malicious activity, system compromise, or policy violations. Its power lies in the fact that all running processes, including heavily obfuscated or encrypted malware, must eventually expose their true nature in memory to execute \cite{bozkir2021catch}. This makes memory analysis particularly potent against fileless malware, which resides primarily in RAM, and polymorphic variants, which may reveal their decoded payload during runtime \cite{sihag2021assistive, hanchenko2024analysis}.

Several studies have highlighted the efficacy of memory forensics. A notable contribution by \citet{bozkir2021catch} demonstrated a powerful synergy between memory forensics and computer vision. Their approach involved capturing memory dumps, transforming these binary data streams into RGB images, and then applying established computer vision feature descriptors – specifically GIST \cite{oliva2001modeling} and HOG \cite{dalal2005histograms} – to extract patterns. These visual features were subsequently fed into machine learning classifiers. Their experiments, conducted on a dataset encompassing 11 classes (including malware families and benign samples), achieved a commendable classification accuracy of 96.39\%. Furthermore, Bozkir et al. pioneered the application of Uniform Manifold Approximation and Projection (UMAP) \cite{mcinnes2018umap} in this context to specifically improve the detection of unknown malware samples not seen during training. Their results showed that using UMAP for dimensionality reduction before classification significantly boosted the detection accuracy for these zero-day threats, with improvements reaching up to 21.83\% \cite{bozkir2021catch}. This work underscored the potential of combining memory analysis with visual pattern recognition and advanced dimensionality reduction for tackling modern, evasive threats.

Complementary research has explored other facets of memory analysis. \citet{nissim2019volatile} focused on the challenge of detecting malware, particularly ransomware, within private cloud environments using memory analysis. They proposed a method based on MinHash, a locality-sensitive hashing technique, to efficiently calculate the similarity between memory dumps. Their approach demonstrated remarkable success, achieving 100\% detection accuracy for ransomware samples within their experimental setup by identifying characteristic memory patterns associated with ransomware activity \cite{nissim2019volatile}. This highlights the utility of similarity analysis techniques applied directly to memory content for identifying specific malware classes known for distinct memory footprints.

Addressing the computational demands sometimes associated with complex feature extraction from large memory dumps, \citet{kara2022fileless} investigated techniques for simplifying the feature engineering process. While acknowledging the power of memory forensics, Kara emphasized the need for approaches that minimize processing overhead, potentially enabling faster analysis or deployment on resource-constrained systems. The research explored various analysis strategies and highlighted ongoing challenges in the field, particularly concerning the evolving tactics of fileless threats \cite{kara2022fileless}.

These studies collectively reinforce the critical importance of memory forensics in modern cybersecurity \cite{sihwail2019malware}. The ability to inspect the runtime state of processes provides a unique vantage point for detecting malware designed explicitly to leave minimal traces on persistent storage \cite{sihag2021assistive, dai2018malware}. Whether through visual pattern recognition \cite{bozkir2021catch}, similarity analysis \cite{nissim2019volatile}, or streamlined feature extraction \cite{kara2022fileless}, analysing volatile memory offers a promising pathway to counter the increasing sophistication of fileless and polymorphic malware.

\subsection{Vision-Based Approaches to Malware Analysis}
\label{subsec:vision_based}

An innovative and increasingly explored avenue for malware analysis involves borrowing techniques from the field of computer vision. The core idea is to represent binary data, such as executable files or memory dumps, as images and then apply image analysis techniques to identify patterns indicative of malicious behaviour or specific malware families \cite{nataraj2011malware, bozkir2021catch}. This visualization approach leverages the human visual system's ability to discern textures and structures, and applies computational methods developed for image recognition to the cybersecurity domain. The underlying hypothesis is that the structural and statistical properties of binary code, when visualized, can create distinct visual patterns that differentiate malware from benign software, and even distinguish between different malware families \cite{nataraj2011malware}.

The seminal work in this area was conducted by \citet{nataraj2011malware}. They pioneered the concept of visualizing malware binaries as grayscale images. In their method, the byte stream of an executable file was interpreted as a sequence of pixel intensities, which were then arranged into a 2D image. They demonstrated that images generated from malware belonging to the same family often exhibited striking visual similarities in texture and layout, which were distinct from images of other families or benign programs. To computationally capture these visual characteristics, they employed the GIST descriptor \cite{oliva2001modeling}, a technique designed to represent the holistic spatial structure of a scene (or, in this case, a malware image). Their experiments showed that classifying malware based on these visual GIST features was not only feasible but surprisingly effective, establishing the foundation for image-based malware classification \cite{nataraj2011malware}. This groundbreaking research opened up possibilities for detecting malware variants, especially polymorphic ones, where traditional signature-based methods might fail due to code modifications that might still preserve underlying structural patterns visible in the image representation \cite{gibert2020rise}.

Building upon this foundation, subsequent research explored various enhancements and alternative techniques within the vision-based paradigm. Recognizing the richness of information available in memory, \citet{zhang2023malware} applied deep learning, specifically Convolutional Neural Networks (CNNs), to analyze visualized memory fragments. Their approach focused on detecting obfuscated malware by learning discriminative features directly from the image representations of memory dumps, achieving a high accuracy of 97.48\% in their experiments \cite{zhang2023malware}. This demonstrated the power of deep learning models to automatically extract relevant visual features, potentially surpassing handcrafted descriptors for certain tasks. Similarly, \citet{vasan2020image} leveraged the power of deep learning by employing transfer learning. They used CNN architectures pre-trained on large-scale image datasets (like ImageNet) and fine-tuned them for the task of malware classification based on image representations. Their IMCEC (Image-based Malware Classification using Ensemble of CNN architectures) approach achieved very high accuracy (99\%) for classifying unpacked malware binaries visualized as images, suggesting that features learned for general image recognition can be effectively adapted for malware analysis \cite{vasan2020image}.

Other researchers explored alternative visualization methods. \citet{yuan2020byte}, for instance, introduced "Markov images" for byte-level malware classification. Instead of directly mapping bytes to pixels, they created images representing the transition probabilities between byte values (treating the byte sequence as a Markov chain). These Markov images capture statistical relationships between byte sequences, offering a different kind of visual signature. Combined with deep learning classifiers, this approach also demonstrated high effectiveness \cite{yuan2020byte}.

These studies collectively illustrate the viability and potential of vision-based techniques in malware analysis. By transforming opaque binary data into a visual format, analysts can leverage powerful image processing and machine learning tools. While early work primarily focused on grayscale images generated from executable files \cite{nataraj2011malware, dai2018malware}, more recent approaches, including the work presented in this thesis and by \citet{bozkir2021catch}, have explored the use of RGB images generated from memory dumps. Representing memory bytes as three-channel RGB pixels potentially allows for encoding richer information compared to single-channel grayscale, offering another dimension for feature extraction and classification, as explored later in this thesis.

\subsection{Feature Extraction Techniques for Malware Classification}
\label{subsec:feature_extraction}

The success of vision-based malware detection hinges critically on the ability to extract meaningful and discriminative features from the generated malware images \cite{gibert2020rise}. These features must effectively capture the underlying structural or textural patterns that differentiate malicious samples from benign ones, or distinguish between various malware families. While deep learning models like CNNs can learn features automatically \cite{zhang2023malware, vasan2020image}, traditional computer vision often relies on handcrafted feature descriptors. Two prominent descriptors utilized in malware visualization research, and central to this thesis, are GIST and Histogram of Oriented Gradients (HOG).

GIST descriptors, introduced by \citet{oliva2001modeling}, are designed to capture the holistic characteristics or the "gist" of an image, representing its overall spatial structure rather than focusing on specific objects or fine details. Inspired by models of human scene perception, GIST aims to provide a low-dimensional representation of the "spatial envelope" of an image \cite{oliva2001modeling}. This is typically achieved by convolving the image with Gabor filters at multiple orientations and scales. The outputs of these filters across different parts of the image are then aggregated (e.g., by averaging within a grid) to produce a compact feature vector that summarizes the dominant spatial frequencies and orientations present in the image \cite{nataraj2011malware, oliva2001modeling}. In the context of malware images, GIST features can capture the global texture and structural layout derived from the binary code patterns, providing a high-level signature of the malware's visual representation \cite{yajamanam2018deep}.

In contrast to the global nature of GIST, the Histogram of Oriented Gradients (HOG) descriptor, popularized by \citet{dalal2005histograms} for human detection tasks, excels at capturing local shape and texture information. The HOG algorithm operates by dividing the image into small connected regions called "cells". For each cell, it computes a histogram of gradient directions (or orientations) for the pixels within that cell. These histograms effectively quantify the dominant edge directions within each local region. To achieve robustness to illumination and shadowing changes, these local histograms are then contrast-normalized by grouping cells into larger spatial blocks and normalizing over these blocks \cite{dalal2005histograms}. The final HOG descriptor is the concatenated vector of these normalized histograms from all blocks. When applied to malware images, HOG features can capture fine-grained local patterns and byte-level structures, such as specific instruction sequences or data patterns that manifest as distinct local textures or edge orientations in the visualized binary data \cite{bozkir2021catch}.

The choice between traditional descriptors and deep learning approaches involves trade-offs. \citet{yajamanam2018deep} explicitly compared the performance of GIST descriptors against deep learning architectures for image-based malware classification. Their findings suggested that while deep learning models offer the advantage of automatic feature learning, eliminating the need for manual feature engineering, well-chosen traditional descriptors like GIST can achieve comparable classification accuracy \cite{yajamanam2018deep}. Furthermore, traditional descriptors often require less computational resources for feature extraction compared to training deep models, which can be advantageous in certain deployment scenarios \cite{yajamanam2018deep}.

Recognizing the different strengths of GIST (global structure) and HOG (local details), researchers, including \citet{bozkir2021catch} and the work presented in this thesis, have explored combining these descriptors. The rationale is that fusing global and local features can create a richer, more comprehensive feature representation that captures a wider range of visual characteristics present in the malware images. This combined feature vector potentially offers superior discriminative power compared to using either GIST or HOG alone, leading to improved classification accuracy. The effectiveness of this feature fusion strategy is a key aspect evaluated in this thesis.

\subsection{Machine Learning for Fileless and Polymorphic Malware Detection}
\label{subsec:ml_detection}

Once relevant features are extracted, whether from memory dumps directly or through visual representations, machine learning (ML) algorithms play a pivotal role in the classification phase \cite{gibert2020rise}. The adaptive and evolving nature of fileless and polymorphic malware renders static, signature-based detection largely inadequate \cite{khoje2024revolutionary}. Machine learning offers a powerful alternative by learning patterns from data to distinguish between malicious and benign samples, even when faced with previously unseen variants or obfuscation techniques \cite{khalid2023insight}.

Several researchers have explored the application of ML specifically for detecting these challenging malware types using memory forensics data. For instance, \citet{khalid2023insight} provided insights into ML-based detection of fileless malware, emphasizing the importance of robust feature extraction from memory artifacts. Their work, along with research by \citet{hossain2024enhanced}, explored various feature engineering techniques suitable for memory dump analysis and investigated the effectiveness of ensemble learning models. Both studies highlighted critical considerations for successful ML application in this domain, such as the necessity of using balanced datasets to prevent classification bias towards the majority class (often benign samples) and the importance of feature normalization to ensure that features with larger numerical ranges do not disproportionately influence the learning process \cite{khalid2023insight, hossain2024enhanced}. These works underscore that careful data preparation and model selection are paramount when applying ML to memory forensics data.

Addressing the specific challenge of polymorphism, \citet{khoje2024revolutionary} focused on identifying structural properties of malware that remain relatively consistent despite code mutations. They proposed using the K-Nearest Neighbors (K-NN) algorithm, a non-parametric instance-based learning method, combined with carefully engineered features representing these stable structural characteristics. Their approach aimed to circumvent the evasion tactics of polymorphic malware by focusing on inherent properties less susceptible to simple obfuscation, achieving high accuracy in their experiments \cite{khoje2024revolutionary}. This demonstrates the value of domain-specific feature engineering tailored to the evasion mechanisms of the targeted malware.

Beyond analyzing memory content or visualized binaries directly, other researchers have applied ML to different data sources reflecting malware behavior. \citet{hammi2024malware} investigated the use of Windows system call sequences for malware detection. By analyzing the sequence of interactions a program makes with the operating system kernel, their ML models aimed to identify patterns characteristic of malicious behavior, which might differ significantly from legitimate software activity \cite{hammi2024malware}. Similarly, \citet{chowdhury2022capturing} explored an alternative representation using ontology-based knowledge graphs. They proposed capturing malware behavior by constructing graphs that represent semantic relationships between different actions, processes, and system resources involved in a malware execution trace. Machine learning could then be applied to these graph structures to classify malware \cite{chowdhury2022capturing}. These studies illustrate the breadth of ML applications in malware detection, extending beyond direct memory or file analysis to encompass runtime behaviors and semantic relationships.

Collectively, these works demonstrate the indispensable role of machine learning in developing adaptive detection mechanisms capable of countering fileless and polymorphic threats. By learning from data, ML models can identify complex patterns associated with malicious activity, offering a more robust defense compared to static signature matching, particularly when dealing with zero-day threats and evasive variants. The choice of features, data sources (memory dumps, system calls, etc.), and ML algorithms remains an active area of research, with different approaches offering varying trade-offs in terms of accuracy, computational cost, and interpretability.

\subsection{Dimension Reduction and Manifold Learning in Malware Analysis}
\label{subsec:dim_reduction}

Feature extraction techniques, particularly when combined (like GIST and HOG in this research), can result in high-dimensional feature vectors. While rich in information, these high-dimensional spaces can pose challenges for machine learning classifiers, often referred to as the "curse of dimensionality". Issues include increased computational complexity, potential overfitting, and the difficulty of visualizing or interpreting the data structure. Dimension reduction techniques aim to mitigate these problems by transforming the data into a lower-dimensional representation while preserving the most important structural information and reducing redundancy \cite{mcinnes2018umap}.

Among various dimension reduction techniques, manifold learning algorithms have shown particular promise for complex datasets like those encountered in malware analysis. These techniques assume that the high-dimensional data actually lies on or near a lower-dimensional manifold embedded within the higher-dimensional space. Manifold learning algorithms attempt to uncover this underlying structure.

A particularly effective and increasingly popular manifold learning technique is Uniform Manifold Approximation and Projection (UMAP), introduced by \citet{mcinnes2018umap}. UMAP is built upon a strong mathematical foundation in Riemannian geometry and algebraic topology. Unlike earlier methods such as t-distributed Stochastic Neighbor Embedding (t-SNE), which primarily focuses on preserving local structure and may struggle with capturing global relationships accurately, UMAP aims to preserve both local and global data structure in the lower-dimensional embedding \cite{mcinnes2018umap, becht2019dimensionality}. It models the data manifold using fuzzy topological structures and then optimizes a low-dimensional representation to match the topology of the high-dimensional data as closely as possible. This balanced preservation of both local neighborhoods and overall global structure makes UMAP particularly well-suited for tasks like malware analysis, where both fine-grained similarities (e.g., within a malware family) and broader relationships (e.g., between different families or between malware and benign samples) are important \cite{ali2019timeCluster}.

The efficacy of manifold learning in the malware domain has been investigated previously. \citet{sharma2018efficacy} evaluated various manifold learning algorithms for clustering malware based on image patterns. Their findings indicated that nonlinear manifold learning methods significantly outperformed traditional linear dimension reduction techniques (like Principal Component Analysis, PCA) in capturing the complex, nonlinear relationships often present in malware data \cite{sharma2018efficacy}.

Building directly on the potential of UMAP, \citet{bozkir2021catch} were the first to specifically apply it to the problem of detecting *unknown* malware based on features extracted from memory images. They demonstrated that reducing the dimensionality of their combined GIST and HOG feature space using UMAP before feeding the features to classifiers led to substantial improvements in identifying malware families that were not part of the training set. As noted earlier, their accuracy improvements reached up to 21.83\% \cite{bozkir2021catch}. This provided strong evidence that UMAP can create more discriminative embeddings that enhance the generalization capabilities of classifiers, making them more effective against zero-day threats.

These studies collectively establish dimension reduction, and specifically nonlinear manifold learning techniques like UMAP, as powerful tools for enhancing malware detection \cite{bozkir2021catch}. By creating more compact and potentially more discriminative feature representations, these methods can improve classifier performance, aid in data visualization, and, crucially, boost the ability to detect previously unseen malware variants – a key challenge in the ongoing fight against cyber threats. The application and optimization of UMAP for unknown malware detection is therefore a significant component of the research presented in this thesis.

\section{Research Gaps}
\label{sec:research_gaps}

The comprehensive review of existing literature reveals significant progress in utilizing memory forensics, computer vision, and machine learning for detecting sophisticated malware. However, several research gaps and challenges remain, which this thesis aims to address:

\begin{itemize}
    \item \textbf{Optimizing Visual Representations and Feature Extraction:} While visualizing malware binaries or memory dumps as images has proven effective \cite{nataraj2011malware, bozkir2021catch}, there is ongoing scope for optimizing this process. Much early work relied on grayscale representations \cite{nataraj2011malware, dai2018malware}. Although RGB representations offer potentially richer information \cite{bozkir2021catch}, further investigation is needed into optimal rendering schemes (e.g., column width choices) and the most effective feature extraction strategies for these color images. Moreover, while individual descriptors like GIST and HOG capture valuable information, relying on only one type may limit performance. There is a need to systematically evaluate the benefits of fusing complementary descriptors (global and local) to create more robust feature sets for improved classification accuracy.
    \item \textbf{Enhancing Unknown Malware Detection:} A persistent challenge in malware detection is identifying novel or zero-day threats that have not been encountered before. While ML models can generalize to some extent, their performance often degrades significantly when faced with entirely new malware families or variants employing unknown evasion techniques. Although initial work applying UMAP showed promise in boosting unknown malware detection \cite{bozkir2021catch}, further research is required to explore optimal UMAP configurations (hyperparameters, distance metrics) and rigorously evaluate its effectiveness across diverse datasets and malware types, particularly in supervised or semi-supervised settings designed to maximize discrimination. Improving the detection rates for unknown samples while minimizing false positives remains a critical gap.
    \item \textbf{Improving Robustness Against Evasion:} Fileless and polymorphic malware constantly evolve their evasion tactics. While memory forensics bypasses many disk-based evasion methods \cite{kara2022fileless}, sophisticated malware might still employ techniques to obfuscate memory artifacts or mimic benign processes. Approaches relying on specific features (visual or otherwise) might be susceptible to adversarial manipulation or future evasion techniques designed to break those specific patterns. Developing detection frameworks that are inherently more robust to minor variations and evolving evasion strategies remains an open challenge.
    \item \textbf{Evaluation on Contemporary Datasets:} The threat landscape changes rapidly. Validating detection techniques requires datasets that include recent and relevant malware samples, encompassing the latest fileless and polymorphic families observed in the wild. Some existing studies may rely on older datasets that do not fully reflect the complexity and diversity of current threats. There is a continuous need for evaluation using up-to-date and challenging datasets to ensure the practical relevance of proposed detection methods.
\end{itemize}

Addressing these gaps is crucial for advancing the state-of-the-art in detecting evasive malware. This thesis contributes to bridging these gaps by proposing an enhanced framework that specifically investigates optimized RGB image representation from memory dumps, evaluates a robust GIST+HOG feature fusion strategy, applies and tunes UMAP explicitly for improved unknown malware detection, and validates the approach on a contemporary dataset featuring challenging fileless and polymorphic malware families. The subsequent chapters will detail the proposed methodology designed to tackle these identified research needs.
