\begin{center}
	\Huge\uppercase{\textbf{Abstract}}
\end{center}

{
\setlength{\baselineskip}{25pt}

This thesis addresses the critical challenge of detecting fileless and polymorphic malware that evades traditional signature-based detection methods. We propose a novel approach integrating memory forensics, computer vision, and machine learning to identify sophisticated threats that operate exclusively in memory or continuously modify their code structure. Our framework converts memory dumps into RGB images and applies a combination of GIST and HOG descriptors to extract distinctive visual features, which are then classified using machine learning algorithms.

Experimental evaluation on the "Poly10" dataset comprising 3,953 memory samples across 11 classes demonstrates superior classification accuracy of 97.49\% using Linear SVM, outperforming baseline methods by 8.26\%. The fusion of GIST and HOG descriptors provides a 3.16\% performance improvement over individual descriptors, validating our multi-scale feature extraction approach. Additionally, we employ Uniform Manifold Approximation and Projection (UMAP) to improve the detection of unknown malware, achieving accuracy improvements up to 36.98\% in binary classification tasks.

With an average processing time of 3.77 seconds per sample, our approach offers a practical solution for detecting advanced malware that bypasses conventional security measures. This research establishes memory visualization as an effective technique for identifying evasive threats, bridging the fields of memory forensics and computer vision to address the evolving challenges in cybersecurity.
\addcontentsline{toc}{chapter}{ABSTRACT}
}
\clearpage
